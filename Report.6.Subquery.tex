% !TEX TS-program = xelatex
% !TEX encoding = UTF-8

% Header {{{1

% This is a simple template for a XeLaTeX document using the "article" class,
% with the fontspec package to easily select fonts.

%\documentclass[11pt]{article} % use larger type; default would be 10pt
\documentclass[12pt]{article}

\usepackage{fontspec} % Font selection for XeLaTeX; see fontspec.pdf for documentation
\defaultfontfeatures{Mapping=tex-text} % to support TeX conventions like ``---''
\usepackage{xunicode} % Unicode support for LaTeX character names (accents, European chars, etc)
\usepackage{xltxtra} % Extra customizations for XeLaTeX

%\setmainfont{Charis SIL} % set the main body font (\textrm), assumes Charis SIL is installed
%\setsansfont{Deja Vu Sans}
%\setmonofont{Deja Vu Mono}

% other LaTeX packages.....
\usepackage{geometry} % See geometry.pdf to learn the layout options. There are lots.
\geometry{a4paper, left=25mm, right=25mm, top=25mm, bottom=25mm} % or letterpaper (US) or a5paper or....
\usepackage[parfill]{parskip} % Activate to begin paragraphs with an empty line rather than an indent

\usepackage{graphicx} % support the \includegraphics command and options
\usepackage{subcaption}

\usepackage{amsmath}
\usepackage{amssymb}

\usepackage{algorithm}
\usepackage{algpseudocode}

\usepackage{listings}
\lstset{basicstyle=\ttfamily,columns=fullflexible,keepspaces=true}

%%%%%%%%%%%%%%%%%%%%%%%%%%%%%%%%%%%%%%%%%%%%%%%%%%

\title{Report 5: GROUP}
\author{Dinh Ngoc Tu}

\begin{document}
\maketitle

%%%%%%%%%%%%%%%%%%%%%%%%%%%%%%%%%%%%%%%%%%%%%%%%%%

\section{Who has the same name as the managers of the ``Finance'' department?}

Query:

\begin{lstlisting}[language=SQL]
select employees.emp_no, employees.last_name
    from employees
    join (
        select employees.emp_no, last_name
            from employees
            join dept_manager
                on employees.emp_no = dept_manager.emp_no
            join departments
                on departments.dept_no = dept_manager.dept_no
            where dept_name = 'Finance') as finance_managers
        on employees.last_name = finance_managers.last_name
        and employees.emp_no != finance_managers.emp_no;
\end{lstlisting}

Output:

\begin{verbatim}
+--------+------------+
| emp_no | last_name  |
+--------+------------+
...
| 493821 | Legleitner |
| 495608 | Legleitner |
| 496062 | Legleitner |
| 497267 | Alpin      |
| 497479 | Alpin      |
| 497947 | Alpin      |
| 497954 | Legleitner |
| 498115 | Alpin      |
| 498634 | Alpin      |
| 499867 | Alpin      |
+--------+------------+
406 rows in set (0.00 sec)
\end{verbatim}

%%%%%%%%%%%%%%%%%%%%%%%%%%%%%%%%%%%%%%%%%%%%%%%%%%

\section{Who in the ``Production'' department were hired after the promotion of the last manager in that department?}

Query:

\begin{lstlisting}[language=SQL]
select employees.emp_no, hire_date
    from employees
    join dept_emp on employees.emp_no = dept_emp.emp_no
    join departments on dept_emp.dept_no = departments.dept_no
    where dept_name = 'Production'
        and hire_date > (
            select max(from_date)
            from dept_manager
            where dept_manager.dept_no = departments.dept_no);
\end{lstlisting}

Output:

\begin{verbatim}
+--------+------------+
| emp_no | hire_date  |
+--------+------------+
...
| 499276 | 1996-11-07 |
| 499296 | 1997-11-13 |
| 499329 | 1996-11-03 |
| 499424 | 1998-01-03 |
| 499459 | 1997-02-03 |
| 499732 | 1997-09-07 |
| 499856 | 1997-05-17 |
| 499916 | 1997-05-18 |
| 499993 | 1997-04-07 |
| 499999 | 1997-11-30 |
+--------+------------+
3758 rows in set (1.46 sec)
\end{verbatim}

%%%%%%%%%%%%%%%%%%%%%%%%%%%%%%%%%%%%%%%%%%%%%%%%%%

\section{Find the average salary of each department, from highest to lowest.}

Query:

\begin{lstlisting}[language=SQL]
select dept_no, avg(salary)
    from (
        select salary, dept_no
            from salaries
            join employees
                on employees.emp_no = salaries.emp_no
            join dept_emp
                on dept_emp.emp_no = employees.emp_no) as salary_list
    group by dept_no
    order by avg(salary) desc;
\end{lstlisting}

Output:

\begin{verbatim}
+---------+-------------+
| dept_no | avg(salary) |
+---------+-------------+
| d007    |  80667.6058 |
| d001    |  71913.2000 |
| d002    |  70489.3649 |
| d008    |  59665.1817 |
| d004    |  59605.4825 |
| d005    |  59478.9012 |
| d009    |  58770.3665 |
| d006    |  57251.2719 |
| d003    |  55574.8794 |
+---------+-------------+
9 rows in set (7.93 sec)
\end{verbatim}

%%%%%%%%%%%%%%%%%%%%%%%%%%%%%%%%%%%%%%%%%%%%%%%%%%

\section{Find the average salary of each job title, from highest to lowest.}

Query:

\begin{lstlisting}[language=SQL]
select title, avg(salary)
    from (
        select title, salary
            from titles
            join employees on employees.emp_no = titles.emp_no
            join salaries on salaries.emp_no = titles.emp_no
            where title like '%Engineer') as titles_list
    group by title
    order by avg(salary) desc;
\end{lstlisting}

Output:

\begin{verbatim}
+--------------------+-------------+
| title              | avg(salary) |
+--------------------+-------------+
| Senior Engineer    |  60543.2191 |
| Engineer           |  59508.0397 |
| Assistant Engineer |  59304.9863 |
+--------------------+-------------+
3 rows in set (7.44 sec)
\end{verbatim}

%%%%%%%%%%%%%%%%%%%%%%%%%%%%%%%%%%%%%%%%%%%%%%%%%%

\end{document}